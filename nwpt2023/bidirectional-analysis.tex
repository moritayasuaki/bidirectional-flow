\documentclass{easychair}
\usepackage{doc}
\usepackage{amssymb}
\usepackage{proof}
\usepackage[all]{xy}
\usepackage{tikz-cd}
\usepackage{amsthm}
\usepackage{amsmath}

\theoremstyle{definition}
\newtheorem{definition}{Definition}[section]


\newcommand{\Pow}{\mathcal{P}}
\newcommand{\Rel}{\mathrm{Rel}}
\newcommand{\Endo}{\mathrm{Endo}}
\newcommand{\Bidir}{\mathrm{Bidir}}
\newcommand{\Unidir}{\mathrm{Unidir}}
\newcommand{\BackConst}{\mathrm{BackConst}}

\newcommand{\tomon}{\to_{\mathrm{mon}}}

\newcommand{\ff}{{f^{\rightarrow}}}
\newcommand{\fb}{{f^{\leftarrow}}}
\newcommand{\gf}{{g^{\rightarrow}}}
\newcommand{\gb}{{g^{\leftarrow}}}
\newcommand{\kf}{{k^{\rightarrow}}}
\newcommand{\kb}{{k^{\leftarrow}}}

\newcommand{\join}{\sqcup}
\newcommand{\bigjoin}{\bigsqcup}
\newcommand{\meet}{\sqcap}
\newcommand{\bigmeet}{\sqcap}

\newcommand{\comp}{\circ}

\title{Bidirectional static analysis and Sound approximation through Galois connections}

\author{
  Dylan McDermott \inst{1}
\and
  Yasuaki Morita  \inst{1}
\and
  Tarmo Uustalu   \inst{1}
}

\institute{
  Reykjavik University,
  Reykjav{\'i}k, Iceland\\
  \email{dylanm@ru.is}, \email{yasuaki20@ru.is}, \email{tarmo@ru.is}
}

\authorrunning{Dylan, Yasuaki, Tarmo}

\begin{document}

\maketitle

\section{Introduction}

We investigate program analysis in particular involving bidirectional dataflow and investigate its composability.
As the traditional data-flow analysis framework does, we consider the domain of dataflow information that captures useful property of programs abstractly and specify how pieces of codes behave on the domain.
The behavior of a piece of code is described by a relation between input (information comes from the upstream of the control flow) and output (information comes from the downstream of the control flow).
These relation can be approximated by another form, for example monotone functions.
Traditional dataflow analysis by lattice and monotone function on CFG is beneficial in terms of these framework can be adapted to wide programs.
In this case, a monotone function on each basic block is defined by forward or backward but not both.
Nonetheless, there are some static analyses involving information flow bidirectionally, such as partial-redundancy elimination [], type inference [], load-pop elimination on stack-machine based language [], and path-sensitive alias analysis [].

\section{Static analysis}
We model static analyses in the style of [Frade 2009]: First we consider
complete lattices whose elements are dataflow values that represent
information on a program point, assign a relation between pre and post
dataflow values for each basic program construct together with reasoning
rules for program compositions such as sequence, choice and loop. We
write $C$, $D$, and $E$ to mean complete lattice of dataflow values, and
represent the type of binary relations on pre and post complete lattices
by subsets of their binary product, e.g, $\Pow(D \times E)$, an d write $S$,
$Q$ and $R$ to mean inhabitants of such types. We conventionally refer
to the first component of binary product (i.e, $D$ in $D \times E$) as a
lattice of pre dataflow values, and the second component of binary
product (i.e, $E$ in $D \times E$) as a lattice of post dataflow values.

And then, we approximate such infomation flow presented by a relation to make it computable.
by a bidirectional transfer function pair: one is a monotone function
called backward function which takes a post dataflow value and returns a
pre dataflow value, and another is called forward transfer function
which takes a pre dataflow value and returns a post data value. A type
of bidirectional transfer function pairs are represented by a product of
two spaces of monotone functions, e.g, $((D \tomon E) \times (E \tomon D))$,
and we write $(\fb, \ff)$, $(\gb , \gf)$ for inhabitants of such types.

\section{Galois connections}
A Galois connection between two posets $(X , \leq)$ and $(Y , \leq)$ is a pair
of monotone functions $F : X \tomon Y$ and $G : Y \tomon X$ such that $F(x) \leq y \iff y \leq G(x)$.
In [Frade 2009], the Galois connection legitimate approximation of a
relation by a pair of function. In this work, we introduce a sequence of
Galois connections factor through classes of representation of
information flow and facilitate comparison of the essential properties
of unidirectional and bidirectional analysis. Fig \ref{fig:galois-connections} shows four representations of information
flow and Galois connections between them: $\Rel(D, E)= (\Pow(D \times E), \subseteq)$
is the poset of binary relations and inclusion order , $\Endo = (D \times E \tomon D \times E, \leq)$
is the poset of endo monotone functions and pointwise order determined
by product order on their codomain. $\Bidir = ((E \tomon D) \times (D \tomon E), \leq)$ is the poset of bidirectional transfer functions and product
order of two pointwise orders. $\Unidir = (E \tomon D , \leq)$ is the poset
of unidirectional transfer functions and pointwise order. Each Galois
connection is given by the following monotone functions (TODO).

\begin{figure}[h]
  \centering
  \begin{tikzcd}
    \Rel(D, E) \\
    \Endo(D, E) \\
    \Bidir(D, E) \\
    \Unidir(D, E)
      \arrow["F"{name=F}, bend right=30, swap, from=1-1, to=2-1]
      \arrow["G"{name=G}, bend right=30, swap, from=2-1, to=1-1]
      \arrow["H"{name=H}, bend right=30, swap, from=2-1, to=3-1]
      \arrow["I"{name=I}, bend right=30, swap, hook, from=3-1, to=2-1]
      \arrow["J"{name=J}, bend right=30, swap, from=3-1, to=4-1]
      \arrow["K"{name=K}, bend right=30, swap, hook, from=4-1, to=3-1]
      \arrow["\dashv", phantom, from=F, to=G]
      \arrow["\dashv", phantom, from=H, to=I]
      \arrow["\dashv", phantom, from=J, to=K]
  \end{tikzcd}
  \label{fig:galois-connections}
  \caption{Galois connections factor through information flow representations}
\end{figure}

By taking an image of a monotone function such that codomain is $\Rel(D, E)$,
we obtain a class of relations (a sub-poset of $\Rel(D, E)$). For
example, $\{ R \in \Rel(D, E) \mid \exists f, R = G(f) \}$ can be obtained from $G$
in the top of Galois connection $F \dashv G$. From a discussion of general
properties about Galois connections, the obtained image is actually the
fixed points of $F \comp G$ and it is one to one correspondence between
the fixed points of $G \comp F$. In terms of information flow, the image
is actual information flow that can be handled by representation of endo
monotone functions. This class of relation is characterized by join-closed relations.
Similarly, by taking an image of $G \comp I$ we obtain a class of relations which is join-closed and butterfly where relation $R$ is buttfly if and only if $(d_{0}, e_{1}) \in R \land (d_{1}, e_{0}) \in R \land d_{0} \leq d \leq d_{1} \land e_{0} \leq e \leq e_{1} \implies (d , e) \in R$.

\section{Program composition}
Furthermore, we use the Galois connection to compose multiple information flows. we find that ordinary relation composition and ... TODO

\end{document}
