\documentclass{easychair}
\usepackage{doc}
\usepackage{amssymb}
\usepackage{proof}
\usepackage[all]{xy}
\usepackage{tikz-cd}
\usepackage{amsthm}
\usepackage{amsmath}
\usepackage{subfigure}

\theoremstyle{definition}
\newtheorem{definition}{Definition}[section]


\newcommand{\Pow}{\mathcal{P}}
\newcommand{\Rel}{\mathrm{Rel}}
\newcommand{\Endo}{\mathrm{Endo}}
\newcommand{\Bidir}{\mathrm{Bidir}}
\newcommand{\Unidir}{\mathrm{Unidir}}
\newcommand{\UnidirConst}{\mathrm{BidirConst}}

\newcommand{\tomon}{\to_{\mathrm{mon}}}

\newcommand{\ff}{{f^{\rightarrow}}}
\newcommand{\fb}{{f^{\leftarrow}}}
\newcommand{\gf}{{g^{\rightarrow}}}
\newcommand{\gb}{{g^{\leftarrow}}}
\newcommand{\kf}{{k^{\rightarrow}}}
\newcommand{\kb}{{k^{\leftarrow}}}

\newcommand{\join}{\sqcup}
\newcommand{\bigjoin}{\bigsqcup}
\newcommand{\meet}{\sqcap}
\newcommand{\bigmeet}{\sqcap}

\newcommand{\comp}{\circ}
\newcommand{\bowtielift}{\mathbin{\overline{\bowtie}}}
\newcommand{\caplift}{\mathbin{\overline{\cap}}}
\newcommand{\rotleq}{\rotatebox[origin=c]{90}{$\leq$}}
\newcommand{\rotsubseteq}{\rotatebox[origin=c]{90}{$\subseteq$}}

%\title{Bidirectional analysis and Sound approximation through Galois connections}

\title{Bidirectional data-flow analyses compared to relational through Galois connections}

\author{
  Dylan McDermott \inst{1}
\and
  Yasuaki Morita  \inst{1}
\and
  Tarmo Uustalu   \inst{1}
}

\authorrunning{McDermott, Morita and Uustalu}

\titlerunning{Bidirectional data-flow analyses}

\institute{
  Reykjavik University,
  Reykjav{\'i}k, Iceland\\
  \email{dylanm@ru.is}, \email{yasuaki20@ru.is}, \email{tarmo@ru.is}
}


\begin{document}

\maketitle

\begin{abstract}
Some data-flow analyses are not backward or forward but bidirectional, data-flow information is propagated in both the backward and forward directions along a program's control-flow. Examples include partial redundancy elimination (in the original form of Morel and Renvoise), type inference for dynamic typing, analyses of stack-based languages such as load-pop reduction, path-sensitive alias analysis.

We model ideal analyses with relations on complete lattices; an analysis is specified by a binary relation for every instruction. We ask how adequately relations can be represented by pairs of backward and forward monotone functions. We conduct this study in terms of Galois connections and their fixed-points. We show that fixed-points of the Galois connection between relations and backward-forward function pairs are precisely relations closed under arbitrary joins with a certain additional property (that we call ``butterfly''). This class of relations is closed under intersection (corresponding to nondeterministic choice), but not closed under relational composition (corresponding to sequencing). Bidirectional analyses from literature involve relations from this class whose composition is not in the class. Accordingly, backward-forward function pairs act soundly as a representation of bidirectional analyses, but this is not perfectly compositional in that one cannot hide the intermediate program points in composition. This is different from the situation with the classical unidirectional analyses.
  
  % We studied bidirectional program analyses and its precision and soundness, and how information flows represented by bidirectional transfer functions are combined in the program composition such as sequential execution and non-deterministic choice. To analyze program analyses meta-mathematically, we associated representations of information flow through Galois connections. We found that bidirectional information flow has property called butterfly and the property is not preserved by sequential composition in general. Finally, we propose two approaches to properly handle sequential composition in bidirectional analysis.
\end{abstract}

\section{Introduction}

We investigate data-flow analyses, specifically bidirectional analyses, and their compositionality.
As is standard in data-flow analysis, we work with a domain of data-flow values describing some aspect of the state of a program's execution at a program point (or possibly different domains for different program points).
The analysis of an instruction is to solve a constraint described by a relation between data-flow values at the entry and the exit of the instruction (for short, pre-values and post-values). We refer to these as \emph{data-flow constraints}.
% The analysis of an instruction is described by a relation between data-flow values, stipulating which data-flow values at the entry and the exit of the instruction (for short, pre-values and post-values) are considered compatible with each other.
Data-flow constraints can be specified---either specified precisely or approximated soundly---by other means, typically monotone functions in the backward (post-to-pre) or forward (pre-to-post) direction, called transfer functions.
Unidirectional data-flow analyses describe a wide class of aspects of program behavior. But some data-flow analyses are bidirectional, they involve information flow in both directions and are specified by a pair of transfer functions for each instruction.
For example, type inference in dynamically typed languages has been specified through relational constraints and unification in Henglein's work \cite{henglein_dynamic_1992}, as well as through the propagation information via bidirectional data-flow functions by Khedker et al \cite{DBLP:journals/cl/KhedkerDM03,frade_bidirectional_2009}.
Bidirectional analyses include partial redundancy elimination in the original formulation of Morel and Renvoise \cite{DBLP:journals/cacm/MorelR79,khedker_bidirectional_1999},load-pop reduction for stack-machine based languages \cite{stack,frade_bidirectional_2009}, path-sensitive alias analysis \cite{jaiswal_bidirectionality_2020}.

\section{Specifying analyses}

We model data-flow analyses in the style of
\cite{frade_bidirectional_2009}.  The domains of data-flow values will
be complete lattices $(D, \leq, \bigjoin)$, $(E, \leq, \bigjoin)$ etc.
% The compatibility relation of data-flow values for a single
% instruction or a bigger program is a relation.
If the data-flow domains for the entry and exit points are $(D, \leq, \bigjoin)$,
$(E, \leq, \bigjoin)$, the corresponding data-flow constraint is a
relation between them, i.e., an element of the set
$\Pow(D \times E)$, which is also a complete lattice; we typically
denote relations by $R$, $S$.  And then we aim to specify such a
data-flow constraint by a pair of monotone backward and forward
transfer functions. The poset of bidirectional transfer function pairs
is the product of two posets of monotone functions, viz.,
$((D \tomon E) \times (E \tomon D))$, and we typically write
$(\fb, \ff)$, $(\gb , \gf)$ for inhabitants of this poset.


% complete lattices whose elements are data-flow values.

% represent
% information on a program point, assign a relation between pre and post
% dataflow values for each basic program construct together with reasoning
% rules for program compositions such as sequence, choice and loop. We
% write $D$, and $E$ for complete lattices of data-flow values, and
% represent the type of binary relations on pre and post complete lattices
% by subsets of their binary product, e.g, $\Pow(D \times E)$, and write $S$,
% $Q$ and $R$ to mean inhabitants of such types. We conventionally refer
% to the first component of binary product (i.e, $D$ in $D \times E$) as a
% lattice of pre dataflow values, and the second component of binary
% product (i.e, $E$ in $D \times E$) as a lattice of post dataflow values.

% And then, we approximate such a compatibility relation by a pair
% of monotone backward and forward transfer functions. The poset
% of bidirectional transfer function pairs is the  product
% of two posets of monotone functions, viz.,
% $((D \tomon E) \times (E \tomon D))$, and we write $(\fb, \ff)$,
% $(\gb , \gf)$ for typical inhabitants of this poset.

A Galois connection between two posets $(X , \leq)$ and $(Y , \leq)$ is a pair
of monotone functions $F : X \tomon Y$ and $G : Y \tomon X$ such that $F(x) \leq y \iff y \leq G(x)$.
In \cite{frade_bidirectional_2009}, a Galois connection between relations and pairs of monotone backward and forward functions was considered.
In this work, we introduce a finer sequence of
Galois connections to facilitate detailed comparison of essential properties
of unidirectional and bidirectional analyses. Fig.~1 shows four specification formats of analyses 
and three Galois connections between them: $\Rel(D, E)= (\Pow(D \times E), \subseteq)$
is the poset of binary relations and inclusion, $\Endo(D , E) = (D \times E \tomon D \times E, {\leq})$
is the poset of monotone functions on pairs and pointwise order determined
by product order on their codomain. $\Bidir(D , E) = ((E \tomon D) \times (D \tomon E), \leq)$ is the poset of pairs of monotone backward and forward functions and product
order of two pointwise orders; $\Unidir(D , E) = (E \tomon D, \leq)$ is the poset of monotone backward functions and pointwise order.

\begin{figure}
  \centering
  \begin{tikzcd}
    \Rel(D, E) & \text{all binary relations} \\
    \Endo(D, E) & \text{all join-closed relations} \\
    \Bidir(D, E) & \text{all join-closed and butterfly relations} \\
    \Unidir(D, E) & \text{all join-closed and pre-down-/post-upclosed relations}
      \arrow["F"{name=F}, bend right=30, swap, from=1-1, to=2-1]
      \arrow["G"{name=G}, bend right=30, swap, from=2-1, to=1-1]
      \arrow["H"{name=H}, bend right=30, swap, from=2-1, to=3-1]
      \arrow["I"{name=I}, bend right=30, swap, hook, from=3-1, to=2-1]
      \arrow["J"{name=J}, bend right=30, swap, from=3-1, to=4-1]
      \arrow["K"{name=K}, bend right=30, swap, hook, from=4-1, to=3-1]
      \arrow["\dashv", phantom, from=F, to=G]
      \arrow["\dashv", phantom, from=H, to=I]
      \arrow["\dashv", phantom, from=J, to=K]
      \arrow["\rotsubseteq", phantom, from=2-2 , to=1-2]
      \arrow["\rotsubseteq", phantom, from=3-2 , to=2-2]
      \arrow["\rotsubseteq", phantom, from=4-2 , to=3-2]
  \end{tikzcd}
  \label{fig:galois-connections}
  \caption{Galois connections between different specification formats of analyses (left) and the corresponding classes of relations (right)}
\end{figure}

The Galois connections are given by the following monotone functions.

\begin{align*}
  F(R) &= \lambda p_{0}. \bigjoin \{p \mid p \leq p_{0} \land p \in R\} &  G(f) &= \{ p \mid p \leq f(p) \} \\
  H(f) &= (\lambda e. \pi_0(f (\top , e)), \lambda d. \pi_1(f (d , \top))) & I(\fb, \ff) &= \lambda (d,e). (\fb (e), \ff (d)) \\
  J(\fb, \ff) &= \fb & K(\fb) &= (\fb , \lambda d. \top)
\end{align*}

By taking the image of a monotone function with codomain $\Rel(D, E)$,
we obtain a class of relations (a subposet of $\Rel(D, E)$). For
example, $\{ R \in \Rel(D, E) \mid \exists f.\, R = G(f) \}$ can be obtained from the right adjoint $G$
of the topmost Galois connection $F \dashv G$. By general
facts about Galois connections, the obtained image is actually the poset of 
fixed-points of $F \comp G$ and is isomorphic to the poset of
fixed-points of $G \comp F$. In terms of specification of analyses, the image
of $G$ consists of those compatibility relations that can be specified by monotone functions on pairs. This class consists of relations that are closed under arbitary joins (for short, join-closed).
Similarly, the image of $G \comp I$ is the class of relations which are join-closed and also enjoy what we call the butterfly property, and the image of $G \comp I \comp K$ is the class of relations which are join-closed and also pre-down-/post-upclosed. We say that a relation $R$ is \emph{join-closed} if $S \subseteq R \implies \bigjoin S \in R$, \emph{butterfly} if $(d_{0}, e_{1}) \in R \land (d_{1}, e_{0}) \in R \land d_{0} \leq d \leq d_{1} \land e_{0} \leq e \leq e_{1} \implies (d , e) \in R$, and \emph{pre-down-/post-upclosed} if $(d_{1}, e_{0}) \in R \land d \leq d_{1} \land e_{0} \leq e \implies (d , e) \in R$. Fig.~2 shows diagrammatic presentations of the butterfly and pre-down-/post-upclosed properties.

\begin{figure}
  \centering
  \begin{tikzcd}
    d_{1} & e_{1} & & d_{1} &   \\
    d & e         & & d     & e \\
    d_{0} & e_{0} & &       & e_{0} \\
      \arrow["\rotleq"{name=dd1}, from=2-1, to=1-1]
      \arrow["\rotleq"{name=d0d}, from=3-1, to=2-1]
      \arrow["\rotleq"{name=ee1}, from=2-2, to=1-2, swap]
      \arrow["\rotleq"{name=e0e}, from=3-2, to=2-2, swap]
      \arrow[""{name=r10}, from=1-1, to=3-2, no head]
      \arrow[""{name=r01}, from=3-1, to=1-2, no head]
      \arrow[""{name=r}, from=2-1, to=2-2, dotted, no head]
      \arrow[""{name=r10}, from=1-1, to=3-2, no head]
      \arrow["\rotleq"{name=dd1'}, from=2-4, to=1-4]
      \arrow["\rotleq"{name=e0e'}, from=3-5, to=2-5, swap]
      \arrow[""{name=r10'}, from=1-4, to=3-5, no head]
      \arrow[""{name=r'}, from=2-4, to=2-5, dotted, no head]
  \end{tikzcd}
  \label{fig:butterfly}
  \caption{Diagrammatic presentations of the butterfly (left) and pre-down-/post-upclosedness (right) conditions}
\end{figure}

\section{Compositionality}

Furthermore, we can use the Galois connections to derive composition operators for different analysis specification formats. We focus on sequential composition and nondeterministic choice. In $\Rel(D, E)$,
the analyses of sequential composition and nondeterministic choice of two programs are given by ordinary composition of binary relations ($\bowtie$) and intersection ($\cap$), respectively. One can lift the operations the $\bowtie$ and $\cap$
into operations on another poset along a Galois connection: in particular for the Galois connection $F \dashv G$, we get operations $f \bowtielift g = F (G(f) \bowtie G(g))$ and $f \caplift g = F(G (f) \cap G(g))$ on $\Endo(D,E)$.
From general facts about Galois connections, the inequations
$G(f) \bowtie G(g) \subseteq G (f \bowtielift g)$ and $G(f) \cap G(g)) \subseteq G(f \caplift g)$ hold for any $f$ and $g$.
This fact ensures that the lifted composition operations are complete wrt.\ their ideal counterpart in $\Rel(D, E)$ (overapproximate them), but does not guarantee their soundness (underapproximation, which is what we really need).
By expanding out the individual definitions of the Galois connections leading to $\Rel(D ,E)$ (namely, $F \dashv G$, $H \comp F \dashv G \comp I$, and $J \comp H \comp F \dashv G \comp I \comp K$) and calculating specifically for each case, we can confirm that all the lifted operations are indeed sound too, except for the lifted sequential composition operator for $\Bidir(D, E)$, namely, $(\fb , \ff) \bowtielift (\gb , \gf) = H(F((G(I(\fb , \ff)) \bowtie G(I(\gb , \gf)))))$. This phenomenon occurs because the butterfly property is not closed under relation composition.

In terms of analyses, this means that the sequential composition of two data-flow constraints, each of them described by a bidirectional transfer function pair, cannot be safely represented by a single bidirectional transfer function pair. This phenomenon does not occur in unidirectional analyses, but only in bidirectional analyses.

In this work, we present two different technical approaches to address this phenomenon.

One approach is to use the fact that soundness of lifted operations still holds for the Galois connection between $n$-ary relations $\Pow(D_{1}, D_{2}, ... , D_{n})$ and $(n-1)$-tuples of pairs of monotone functions $((D_{1} \tomon D_{2}) \times (D_{2} \tomon D_{1}) \times ... \times (D_{n-1} \tomon D_{n}) \times (D_{n} \tomon D_{n-1}))$. In this approach, compositionality is somewhat broken in that intermediate program points cannot be hidden in sequential composition. 

Another approach is to restrict the class of relations to a class narrower than the class of join-closed butterfly relations but closed under relation composition and intersection. We show that the poset $\UnidirConst(D, E) = ((E \tomon D) \times E, \leq)$ (with forward functions constant) locates between $\Unidir(D , E)$ and $\Bidir(D, E)$, the class of relations corresponding to $\UnidirConst(D, E)$ is  closed under both $\bowtie$ and $\cap$, and the lifted operations are sound.
This approach does not always allow one to capture the original bidirectional analysis exactly. For example, in the load-pop reduction analysis, the data-flow constaints for all single instructions are butterfly, but those for some sequences (e.g., \textsf{load}; \textsf{dup}) are not. Still, this approach can provide a reasonable systematic overapproximation of the original analysis in some cases.

\paragraph{Acknowledgements} This research was supported by the
Icelandic Research Fund grant 228684-052.

\bibliographystyle{plain}
\bibliography{bidirectional-analysis}
\end{document}
