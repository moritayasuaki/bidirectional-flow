\documentclass[9pt]{article}

\usepackage[top=2.5cm,bottom=2.5cm,left=2.5cm,right=2.5cm,a4paper]{geometry}

\usepackage{amssymb}

\usepackage{proof}
\usepackage[all]{xy}


\newcommand{\Pow}{\mathcal{P}}
\newcommand{\tomon}{\to_{\mathrm{mon}}}

\newcommand{\ff}{{f^{\rightarrow}}}
\newcommand{\fb}{{f^{\leftarrow}}}
\newcommand{\gf}{{g^{\rightarrow}}}
\newcommand{\gb}{{g^{\leftarrow}}}
\newcommand{\kf}{{k^{\rightarrow}}}
\newcommand{\kb}{{k^{\leftarrow}}}

\newcommand{\join}{\sqcup}
\newcommand{\bigjoin}{\bigsqcup}
\newcommand{\meet}{\sqcap}
\newcommand{\bigmeet}{\sqcap}  % there is no \bigsqcap ??

\newcommand{\con}{\wedge}
\newcommand{\imp}{\to}
\newcommand{\equ}{\leftrightarrow}

\newcommand{\fst}{\mathsf{fst}}
\newcommand{\snd}{\mathsf{snd}}

\newcommand{\prC}{\mathsf{pr}_C}
\newcommand{\prD}{\mathsf{pr}_D}
\newcommand{\prE}{\mathsf{pr}_E}
\newcommand{\prCD}{\mathsf{pr}_{C \times D}}
\newcommand{\prDE}{\mathsf{pr}_{D \times E}}
\newcommand{\prCE}{\mathsf{pr}_{C \times E}}

\newcommand{\caplift}{\mathbin{\overline{\cap}}}

\newcommand{\bowtielift}{\mathbin{\overline{\bowtie}}}


\begin{document}

Assume $D$, $E$ have arbitrary joins, i.e., are complete lattices. 

\[
\xymatrix@R=5pc@C=0.3pc{  
  (\Pow(D \times E), \subseteq) \ar@/_1pc/[rrr]_{F} \ar@{}[rrr]|{\top}
  \ar@/_1pc/[d]_{F_0}  \ar@{}[d]|{\dashv}
& & & (D \times E \tomon D \times E, \leq) \ar@/_1pc/[rrr]_{H} \ar@{}[rrr]|{\top}
\ar@/_1pc/[lll]_{G}
\ar@/_1pc/[dl]_{G_0} \ar@{}[dl]|{\vdash}
\ar@/_1pc/[dr]_{H_0} \ar@{}[dr]|{\dashv}
& & & ((E \tomon D) \times (D \tomon E), \leq) \ar@/_1pc/[lll]_{I}
   \ar@{=}[d] \\
\textrm{join-closed} \ar@/_1pc/[rr] \ar@{}[rr]|{\cong}
        \ar@/_1pc/[d]_{H_{00}} \ar@{}[d]|{\dashv} \ar@{(->}@/_1pc/[u]_{U}
& & \textrm{coclosure ops} \ar@/_1pc/[ll]  \ar@{(->}@/_1pc/[ur]_{U}
    & & \textrm{bidir} \ar@/_1pc/[rr] \ar@{}[rr]|{\cong}  \ar@{(->}@/_1pc/[ul]_U
    & &  \ar@/_1pc/[ll] \ar@/_1pc/[d]_{G_{00}}  \ar@{}[d]|{\vdash} \\
\textrm{join-closed butterflies} \ar@/_1pc/[rrrrrr] \ar@{}[rrrrrr]|{\cong}
         \ar@{(->}@/_1pc/[u]_{U}
& & & & & & \textrm{nice} \ar@/_1pc/[llllll]
  \ar@{(->}@/_1pc/[u]_{U}
}
\]  



\begin{eqnarray*}
  R \textrm{~join-closed}
  & \equ & \forall S.\, S \subseteq R \imp \bigjoin S \in R \\
  R \textrm{~a butterfly}
  & \equ & \forall d_0,e_0,d,d',e,e'.\, \\
  & & \qquad
        d \leq d_0 \leq d' \con e \leq e_0 \leq e'
           \con (d,e') \in R \con (d',e) \in R \imp (d_0,e_0) \in R \\
  f \textrm{a coclosure op}
  & \equ & \forall p.\, f(p) \leq p \con f(p) \leq f(f(p)) \\
  f \textrm{~bidir}
  & \equ & \forall d,e.\, (\fst(f(\top,e)),\snd(f(d,\top))) \leq f(d,e) \\
  (\fb,\ff) \textrm{~nice}
  & \equ & (\forall e.\, \fb(e) \leq \fb(e \meet \ff(\fb(e))))
           \con (\forall d.\, \ff(d) \leq \ff(d \meet \fb(\ff(d))))
\end{eqnarray*}  

\begin{eqnarray*}
  F(R) & = &
    \lambda p_0.\, \bigjoin \{p \mid p \leq p_0 \con p \in R\} \\        
  G(f) & = & \{p \mid p \leq f(p)\} \\
  H(f) & = &
    (\lambda e.\, \fst(f(\top,e)), \lambda d.\, \snd(f(d,\top))) \\         
  I(\fb,\ff) & = & \lambda (d,e).\, (\fb(e),\ff(d)) \\
  HF(R) & = &
    (\lambda e_0.\, \bigjoin\{d \mid \exists e.\, e \leq e_0 \con (d,e) \in R\},
    \lambda d_0.\, \bigjoin \{e \mid \exists d.\, d \leq d_0 \con (d,e) \in R\}) \\
  GI(\fb,\ff) & = &
                    \{ (d,e) \mid d \leq \fb(e) \con e \leq \ff(d)\} 
\end{eqnarray*}

\begin{eqnarray*}
  F_0(R) & = & \{\bigjoin S \mid S \subseteq R\} \\
  G_0(f) & = & \lambda p_0.\, \bigjoin \{ p \mid p \leq p_0 \con p \leq f(p) \}
               = \lambda p_0.\, \nu p.\, p_0 \meet f(p) \\
  H_{00}(R) & = & \{ (d,e) \mid \exists d_0,d_1,e_0,e_1.\,
        d_0 \leq d \leq d_1 \con e_0 \leq e \leq e_1
           \con (d_0,e_1) \in R \con (d_1,e_0) \in R\} \\
  H_0 (f) & = & \lambda (d,e).\, (\fst(f(\top,e)), \snd(f(d,\top))) \\
  G_{00} (\fb,\ff) & = & (\lambda e_0.\, \bigjoin \{d \mid \exists e.\, e \leq e_0 \con d \leq \fb(e) \con e \leq \ff(d) \},
                         \lambda d_0.\, \bigjoin \{e \mid \exists d.\, d \leq d_0 \con d \leq \fb(e) \con e \leq \ff(d) \})
\end{eqnarray*}


The functions $F$, $G$ etc are all order-preserving, i.e. poset maps. 

Proof of $F \dashv G$:

\[
\infer={R \subseteq G(f)}{
  \infer=[(i)]{\forall p.\, p \in R \imp p \leq f(p)}{
    \infer={\forall p_0,p.\, p \leq p_0 \con p \in R \imp p \leq f(p_0)}{
      \infer={\forall p_0.\, \bigjoin \{p \mid p \leq p_0 \con p \in R\} \leq f(p_0)}{
        F(R) \leq f
      }
    }
  }
}  
\]

(i):$\downarrow$: Instantiate $p_0 := p$.

(i):$\uparrow$: Use monotonicity of $f$ to conclude $f(p) \leq f(p_0)$ from
$p \leq p_0$.

\bigskip


Proof of $H \dashv I$:

\[
\infer={f \leq I(\fb,\ff)}{
  \infer={\forall d,e.\, f(d,e) \leq (\fb(e),\ff(d))}{
    \infer=[(ii)]{(\forall d,e.\, \fst(f(d,e)) \leq \fb(e))
        \con (\forall d,e.\, \snd(f(d,e)) \leq \ff(d))}{
      \infer={(\forall e.\, \fst(f(\top,e)) \leq \fb(e))
        \con (\forall d.\, \snd(f(d,\top)) \leq \ff(d))}{
       H(f) \leq (\fb,\ff) 
      }
    }  
  }
}  
\]  

(ii):$\downarrow$: Use monotonicity of $f$ to conclude
$f(d,e) \leq f(\top,e)$ and $f(d,e) \leq f(d, \top)$.

(ii):$\uparrow$: Instantiate $d := \top$ and $e := \top$.

\bigskip

Proof of ``$R$ a fixpoint of $GF$ $\equ$ $R$ join-closed'':

\[
\infer={R \textrm{~join-closed}}{
  \infer=[(iii)]{\forall S.\, S \subseteq R \imp \bigjoin S \in R}{    
    \infer={\forall p_0.\, p_0 \leq \bigjoin \{p \mid p \leq p_0 \con p \in R\} \imp p_0 \in R}{
      \infer={GF(R) \subseteq R}{
        R \textrm{~a fixpoint of $GF$}
      }  
    }
  }
}  
\]

(iii):$\downarrow$: Instantiate $p_0 := \bigjoin S$. From
$S \subseteq R$, using that $\forall p.\, p \in S \imp p \leq \bigjoin S$,
we have $S \subseteq \{p \mid p \leq \bigjoin S \con p \in
R\}$. Hence
$\bigjoin S \leq \bigjoin \{p \mid p \leq \bigjoin S \con p \in R\}$.

(iii):$\uparrow$: Instantiate
$S := \{p \mid p \leq p_0 \con p \in R\}$.  From
$p_0 \leq \bigjoin \{p \mid p \leq p_0 \con p \in R\}$, using that
$\bigjoin \{p \mid p \leq p_0 \con p \in R\} \leq p_0$,
we have $p_0 = \bigjoin \{p \mid p \leq p_0 \con p \in R\}$.

\bigskip

Proof of ``$f$ a fixpoint of $FG$ $\equ$ $f$ a coclosure operator'':

\[
\infer={f \textrm{~a coclosure operator}}{
  \infer=[(iv)]{\forall p_0.\, f(p_0) \leq p_0 \con f(p_0) \leq f(f(p_0))}{
    \infer={\forall p_0, p_*.\, (\forall p.\, p \leq p_0 \con p \leq f(p) \imp p \leq p_*) \imp f(p_0) \leq p_*}{
      \infer={\forall p_0.\, f(p_0) \leq \bigjoin \{p \mid p \leq p_0 \con p \leq f(p)\}}{
        \infer={f \leq FG(f)}{
           f \textrm{~a fixpoint of $FG$}
        }
      }
    }
  }
}  
\]

(iv):$\downarrow$: Instantiate $p_* := p_0 \meet f(f(p_0))$. Use
monotonicity of $f$ several times to conclude $p \leq f(p) \leq f(f(p)) \leq  f(f(p_0))$ from
$p \leq p_0$ and $p \leq f(p)$ .

(iv):$\uparrow$: Instantiate $p := f(p_0)$.

\bigskip

Proof of ``$f$ a fixpoint of $IH$ $\equ$ $f$ bidir'':

\[
\infer={f \textrm{~bidir}}{
  \infer={\forall d,e.\, (\fst(f(\top,e)),\snd(f(d,\top))) \leq f(d,e)}{
    \infer={IH(f) \leq f}{
      f \textrm{~a fixpoint of $IH$}}{
    }
  }
}
\]


\bigskip

Proof of ``$(\fb,\ff)$ a fixpoint of $HI$ always'':

\[
\infer={\textrm{true}}{
  \infer={(\forall e.\, \fb(e) \leq \fst(\fb(e),\ff(\top))) 
    \con (\forall d.\, \ff(d) \leq \snd(\fb(\top),\ff(d)))}{
    \infer={(\fb,\ff) \leq HI(\fb,\ff)}{
    (\fb,\ff) \textrm{~a fixpoint of $HI$}
    }
  }
}  
\]  

\bigskip


Proof of ``$R$ a fixpoint of $GIHF$ $\equ$ $R$ is a join-closed butterfly'':

\[
\infer={R \textrm{~join-closed} \con R \textrm{~a butterfly}}{
  \infer=[(v)]{R \textrm{~join-closed} \con (\forall d_0,e_0,d,d',e,e'.\, 
        d \leq d_0 \leq d' \con e \leq e_0 \leq e'
           \con (d,e') \in R \con (d',e) \in R \imp (d_0,e_0) \in R)}{    
           \infer={R \textrm{~join-closed} \con (\forall d_0,e_0.\, d_0 \leq \bigjoin \{d' \mid \exists e.\, e \leq e_0 \con (d',e) \in R\} \con e_0 \leq \bigjoin \{e' \mid \exists d.\, d \leq d_0 \con (d,e') \in R\} \imp (d_0,e_0) \in R)}{
        \infer=[(a)]{GF(R) \subseteq R \con GIFH(R) \subseteq R}{        
          \infer={GIHF(R) \subseteq R}{
            R \textrm{~a fixpoint of $GIHF$}
          }  
        }
     }  
  }
}  
\]

(a) Notice that $\forall f.\, f \leq IH(f)$. 

(v):$\downarrow$: Conclude
$d_0 \leq \bigjoin \{d' \mid \exists e.\, e \leq e_0 \con (d',e)
\in R\}$ from $d_0 \leq d'$, $e \leq e_0$ and $(d',e) \in R$.
Similarly conclude
$e_0 \leq \bigjoin \{e' \mid \exists d.\, d \leq d_0 \con (d,e')
\in R\}$ from $e_0 \leq e'$, $d \leq d_0$ and $(d, d') \in R$.

\newcommand{\Sb}{S^\leftarrow}
\newcommand{\Sf}{S^\rightarrow}

(v):$\uparrow$: Let
$\Sb(e_0) := \{(d',e) \mid e \leq e_0 \con (d', e) \in R\}$. From
$d_0 \leq \bigjoin \{d' \mid \exists e.\, e \leq e_0 \con (d',e)
\in R\}$, using that
$\bigjoin \{d' \mid \exists e.\, e \leq e_0 \con (d',e) \in R\}
= \fst (\bigjoin \{(d',e) \mid e \leq e_0 \con (d', e) \in
R\})$, conclude $d_0 \leq \fst(\bigjoin \Sb(e_0))$. It is evident that
$\snd(\bigjoin \Sb(e_0)) \leq e_0$. From join-closedness of $R$, noticing that
$\Sb(e_0) \subseteq R$, conclude $\bigjoin \Sb(e_0) \in R$.

Let also $\Sf(d_0) := \{(d,e') \mid d \leq d_0 \con (d, e') \in R\}$.
Conclude that $e_0 \leq \snd(\bigjoin \Sf(d_0))$, $\fst(\bigjoin \Sf(d_0)) \leq d_0$ and
$\bigjoin \Sf(d_0) \in R$.

Now instantiate $(d',e) := \bigjoin \Sb(e_0)$. We get $d_0 \leq d'$, $e
\leq e_0$ and $(d',e) \in R$.

Also instantiate $(d, e') := \bigjoin \Sf(d_0)$ and get $e_0 \leq
e'$, $d \leq d_0$ and $(d, e') \in R$. 


\bigskip


Proof of ``$(\fb,\ff)$ a fixpoint $HFGI$ $\equ$ $(\fb,\ff)$ nice'':

\[
\infer={(\fb, \ff) \textrm{~nice}}{
  \infer={\forall e.\, \fb(e) \leq \fb(e \meet \ff(\fb(e)))) \con \textrm{symm cond}}{
    \infer=[(vi)]{(\forall d_*.\, (\forall d,e.\, e \leq e_0 \con d \leq \fb(e) \con e \leq \ff(d) \imp d \leq d_*) \imp \fb(e_0) \leq d_*) \con \textrm{symm cond}}{
      \infer={\fb(e_0) \leq \bigjoin \{d \mid \exists e.\, e \leq e_0 \con d \leq \fb(e) \con e \leq \ff(d)\} \con \textrm{symm cond}}{
        \infer={(\fb,\ff) \leq HFGI(\fb,\ff)}{
          (\fb,\ff) \textrm{~a fixpoint of $HFGI$}
        }
      }
    }
  }
}  
\]  

(vi):$\downarrow$: Instantiate $d_* := \fb(e_0 \meet \ff(\fb(e_0)))$.
Use monotonicity of $\fb$ and $\ff$ to conclude
$d \leq \fb(e) \leq \fb(e_0 \meet \ff(d)) \leq
\fb(e_0 \meet \ff(\fb(e)) \leq \fb(e_0 \meet \ff(\fb(e_0)))$ from $e \leq e_0$,
$d \leq \fb(e)$ and $e \leq \ff(d)$.

(vi):$\uparrow$: Instantiate $d := \fb(e_0)$,
$e := e_0 \meet \ff(\fb(e_0))$.




\[
  \scriptsize
\hspace*{-1.5cm}  
\xymatrix@R=5pc@C=0.1pc{  
  (\Pow(D \times E), \subseteq) \ar@/_1pc/[rrr]_{F} \ar@{}[rrr]|{\top}
  \ar@/_1pc/[d]_{F_0}  \ar@{}[d]|{\dashv}
& & & (D \times E \tomon D \times E, \leq) \ar@/_1pc/[rrr]_{H} \ar@{}[rrr]|{\top}
\ar@/_1pc/[lll]_{G}
\ar@/_1pc/[dl]_{G_0} \ar@{}[dl]|{\vdash}
\ar@/_1pc/[dr]_{H_0} \ar@{}[dr]|{\dashv}
      & & & ((E \tomon D) \times (D \tomon E), \leq)
      \ar@/_1pc/[lll]_{I} \ar@{=}[dl]
      \ar@/_1pc/[rrr]_{J} \ar@{}[rrr]|{\top}
       \ar@/_1pc/[dr]_{J_0} \ar@{}[dr]|{\dashv}
             & & & (E \tomon D, \leq) \ar@{=}[d]
                \ar@/_1pc/[lll]_{K} \\
\textrm{join-closed} \ar@/_1pc/[rr] \ar@{}[rr]|{\cong}
        \ar@/_1pc/[d]_{H_{00}} \ar@{}[d]|{\dashv} \ar@{(->}@/_1pc/[u]_{U}      & & \textrm{coclosure ops} \ar@/_1pc/[ll]  \ar@{(->}@/_1pc/[ur]_{U}   
    & & \textrm{bidir} \ar@/_1pc/[r] \ar@{}[r]|{\cong}  \ar@{(->}@/_1pc/[ul]_U
   \ar@/_1pc/[dr]_{J_{00}} \ar@{}[dr]|{\dashv}
        & \ar@/_1pc/[l] 
          & & \textrm{unidir} \ar@/_1pc/[rr] \ar@{}[rr]|{\cong}  \ar@{(->}@/_1pc/[ul]_U
          & & \ar@/_1pc/[ll] \ar@{=}[d]\\
\textrm{join-closed butterflies} 
    \ar@/_1pc/[d]_{J_{000}} \ar@{}[d]|{\dashv} \ar@{(->}@/_1pc/[u]_{U}
& & 
    & & & \textrm{unidir'} \ar@{(->}@/_1pc/[ul]_{U}
            \ar@/_1pc/[rrrr] \ar@{}[rrrr]|{\cong}
          & 
            & & & \ar@{=}[d] \ar@/_1pc/[llll] \\
\textrm{join-closed down-up-closed} \ar@{(->}@/_1pc/[u]_{U} \ar@/_1pc/[rrrrrrrrr] \ar@{}[rrrrrrrrr]|{\top}
& & & & & & & & &  \ar@/_1pc/[lllllllll]            
}
\]  

 

\begin{eqnarray*}
  R \textrm{~down-up-closed}
  & \equ & \forall d_0,e_0,d',e.\, \\
  & & \qquad
        d_0 \leq d' \con e \leq e_0
           \con (d',e) \in R \imp (d_0,e_0) \in R \\
  f \textrm{~unidir'}
  & \equ & \forall d,e.\, (\fst(f(\top,e)),\top) \leq f(d,e) \\
  (\fb,\ff) \textrm{~unidir}
  & \equ & \forall e.\ \top \leq \ff(e)
\end{eqnarray*}           

\begin{eqnarray*}
  J(\fb,\ff)
  & = & \fb \\
  K(\fb)
  & = & (\fb, \lambda d.\, \top) \\  
  J_0(\fb,\ff)
  & = & (\fb, \lambda d.\, \top) \\
  J_{00}(f) 
  & = & \lambda p.\ (\fst(f(p)), \top) \\
  J_{000}(R)
  & = & \{ (d_0,e_0) \mid \exists d',e.\
          d_0 \leq d' \con e \leq e_0 \con (d',e) \in R \}
\end{eqnarray*}

\begin{eqnarray*}
 JHF(R) & = &
    \lambda e_0.\, \bigjoin\{d \mid \exists e.\, e \leq e_0 \con (d,e) \in R\} \\
 GIK(\fb) & = &
                    \{ (d,e) \mid d \leq \fb(e) \} 
\end{eqnarray*}

Proof of ``$R$ a fixpoint of $GIKJHF$ $\equ$ $R$ is join-closed and down-up-closed '':

\[
\infer={R \textrm{~join-closed} \con R \textrm{~down-up-closed}}{
  \infer=[(v)]{R \textrm{~join-closed} \con (\forall d_0,e_0,d',e.\, 
        d_0 \leq d' \con e \leq e_0 
           \con (d',e) \in R \imp (d_0,e_0) \in R)}{    
           \infer={R \textrm{~join-closed} \con (\forall d_0,e_0.\, d_0 \leq \bigjoin \{d' \mid \exists e.\, e \leq e_0 \con (d',e) \in R\}
                \imp (d_0,e_0) \in R)}{
        \infer=[(a)]{GF(R) \subseteq R \con GIKJFH(R) \subseteq R}{        
          \infer={GIKJHF(R) \subseteq R}{
            R \textrm{~a fixpoint of $GIHF$}
          }  
        }
     }  
  }
}  
\]

(a) Notice that $\forall f.\, f \leq IH(f)$. 

(v):$\downarrow$: Conclude
$d_0 \leq \bigjoin \{d' \mid \exists e.\, e \leq e_0 \con (d',e)
\in R\}$ from $d_0 \leq d'$, $e \leq e_0$ and $(d',e) \in R$.

%\newcommand{\Sb}{S^\leftarrow}

(v):$\uparrow$: Let
$\Sb(e_0) := \{(d',e) \mid e \leq e_0 \con (d', e) \in R\}$. From
$d_0 \leq \bigjoin \{d' \mid \exists e.\, e \leq e_0 \con (d',e)
\in R\}$, using that
$\bigjoin \{d' \mid \exists e.\, e \leq e_0 \con (d',e) \in R\}
= \fst (\bigjoin \{(d',e) \mid e \leq e_0 \con (d', e) \in
R\})$, conclude $d_0 \leq \fst(\bigjoin \Sb(e_0))$. It is evident that
$\snd(\bigjoin \Sb(e_0)) \leq e_0$. From join-closedness of $R$, noticing that
$\Sb(e_0) \subseteq R$, conclude $\bigjoin \Sb(e_0) \in R$.

Now instantiate $(d',e) := \bigjoin \Sb(e_0)$. We get $d_0 \leq d'$, $e
\leq e_0$ and $(d',e) \in R$.




\subsection*{What is analysis about?}

How does analysis work with any of the data $R$, $f$, $(\fb,\ff)$ or $\fb$?

Given $p_0$, we are after $p$ such that $p \leq p_0 \con p \in
R$. \emph{If} $R$ is \emph{join-closed}, then there is the greatest one
among them, which is this is
$F(R)(p_0) = \bigjoin \{ p \mid p \leq p_0 \con p \in R\}$.

Or, given $p_0$, we are after the greatest $p$ such that
$p \leq p_0 \con p \leq f(p)$. This is
$FG(f)(p_0) = \bigjoin \{ p \mid p \leq p_0 \con p \leq f(p) \} = \nu p.\, p_0
\meet f(p)$, \emph{without assumptions}.  If $D \times E$ has no
infinite descending chains, then this can be computed as
$\bigmeet f_i$ where $f_0 = \top$ and $f_{i+1} = p_0 \meet
f(f_i)$. For all $i$, it is the case that $f_{i+1} \leq f_i$. One has
$\bigmeet f_i = f_{i_0}$ where $i_0$ is the smallest $i$ such that
$f_i \leq f_{i+1}$.

Or, given $(d_0,e_0)$, we are after the greatest $(d,e)$ such that
$(d,e) \leq (d_0,e_0) \con d \leq \fb(e) \con e \leq \ff(d)$.  This is
$HFGI(\fb,\ff)(d_0,e_0) = \bigjoin \{(d,e) \mid (d,e) \leq (d_0,e_0) \con d \leq \fb(e) \con e \leq
\ff(d)\} = \nu (d,e).\, (d_0 \meet \fb(e),e_0 \meet \ff(d))$.

In the case of a unidirectional analysis, given $(d_0,e_0)$, where
we typically take $d_0 = \top$, we are after the greatest $(d,e)$ such
that $(d,e) \leq (d_0,e_0) \con d \leq \fb(e)$.
Here the answer is simply $(d_0 \meet \fb(e_0), e_0)$.

\subsection*{Nondeterminism}

\paragraph{Relations}

On $(\Pow(D \times E), \subseteq)$ we have a symmetric monoidal and
idempotent operator $\cap$. This is our reference interpretation
of nondeterminism on our reference model.

The operator $\cap$ restricts to the poset of join-closed relations:
If $R$, $R'$ are join-closed, then $R \cap R'$ is also join closed.

Proof: Suppose $S \subseteq R \cap R'$, then $S \subseteq R$ and
$S \subseteq R'$.  By join-closedness of $R$ and $R'$,
$\bigjoin S \in R$ and $\bigjoin S \in R'$.  Hence
$\bigjoin S \in R \cap R'$.

\paragraph{Functions on pairs}

The operator $\cap$ induces an operator $\caplift$ on
$(D\times E \tomon D \times E, \leq)$ via
\[
f \caplift g = F (G (f) \cap G(g))
\]  
This construction is general, it only uses that $F$ and $G$ are functors
from and to $\Pow(D\times E, \subseteq)$.

A general argument (only using $F \dashv G$) shows that $F$ is oplax
monoidal and $G$ is lax monoidal. Indeed,
\[
F (R \cap S) \leq F (GF (R) \cap GF (S) = F (R) \caplift F (S)
\]  
using the unit and monotonicity of $F$ and $\cap$, and also
\[
G (f) \cap G (g) \leq  GF (G(f) \cap G(g)) =  G (f \caplift g)
\]  
using the counit.

But in fact $G$ in our specific adjunction is also oplax monoidal, so
altogether it is monoidal.  This uses that a relation is in the image
of $G$ iff it is join-closed and that $\cap$ takes join-closed
relations to join-closed relations.

Proof:
\begin{eqnarray*}
\lefteqn{ G (f \caplift g) } \\
  & = & GF (G(f) \cap G (g)) \\
  & & \quad (i) \\
  & = & GFG (h) \\
  & & \quad \textrm{~unit and monotonicity of $G$} \\
& \subseteq & G (h) \\
& = & G (f) \cap  G (g) 
\end{eqnarray*}  

(i): $G (f)$ and $G (g)$ are in the image of $G$, therefore
join-closed.  Therefore $G(f) \cap G(g)$ is also join-closed.  But
then $G(f) \cap G(g)$ is in the image of $G$, i.e., there must exist
$h$ such that $G(h) = G(f) \cap G(g)$.

Oplax monoidality of $G$, i.e., the fact that 
\[
G (f \caplift g) \subseteq G (f) \cap G (g)  
\]
means that $\caplift$ is a sound approximation of $\cap$: using first
$\caplift$ on the imperfect analyses $f$ and $g$ and then $G$ gives a
less precise but sound result than using first $G$ and then $\cap$.

Monoidality of $G$ means that $\caplift$ is in fact as precise as
$\cap$.

Explicitly, $\caplift$ is the operation
\begin{eqnarray*}
f \caplift g & = & F (G (f) \cap G(g)) \\
& = & \lambda p_0.\, \bigjoin \{ p \mid p \leq p_0 \con p \in G (f) \cap G (g) \} \\
& = & \lambda p_0.\, \bigjoin \{ p \mid p \leq p_0 \con p \leq f (p) \con p \leq g (p) \} \\
& = & \lambda p_0.\, \nu p.\, p_0 \meet f(p) \meet g(p)
\end{eqnarray*}
We notice that this involves a fixpoint computation. 

As an alternative to $\caplift$, we can use some other monoidal
operation on $(D \times E \tomon D \times E, \leq$ for which $G$ is
oplax monoidal.

It turns out that we can just use $\meet$ (pointwise meet). $G$ wrt.\
$\meet$ is oplax monoidal.  Indeed, we have
$G (f \meet g) \subseteq G (f)$ and $G (f \meet g) \subseteq G (f)$
from monotonicity of $G$, therefore
$G (f \meet g) \subseteq G (f) \cap G (g)$.

So $\meet$ gives us a sound approximation of $\cap$.

Moreover $G$ wrt.\ $\meet$ is even lax monoidal, thus monoidal.

Proof: Write $k$ for $f \meet g$. We have
\begin{eqnarray*}
  f \caplift g
& = & \lambda p_0.\, \nu p.\, p_0 \meet k (p) \\
& = & \lambda p_0.\, p_0 \meet k (\nu p.\, p_0 \meet k (p)) \\
& \leq & \lambda p_0.\, k (\nu p.\, p_0 \meet k (p)) \\
& = & \lambda p_0.\, k (p_0 \meet k (\nu p.\, p_0 \meet k (p))) \\
& \leq & \lambda p_0.\, k (p_0) \\
& = & f \meet g
\end{eqnarray*}
By monotonicity of $G$ and by lax monoidality of $G$ wrt.\ $\caplift$
(which we proved before), it follows that
$G (f) \cap G (g) \subseteq G (f \caplift g) \subseteq G (f \meet g)$.

We note that $f \meet g$ is as precise as $f \caplift g$ as an
approximation, but more efficient. 

% \vspace*{5cm}

\paragraph{Backward and forward functions}

For $((E \tomon D) \times (D \tomon E), \leq$ and the poset maps $HF$
and $GI$ similar considerations apply.

We can construct
\begin{eqnarray*}
\lefteqn{  (\fb, \ff) \caplift (\gb, \gf) } \\
  & = & HF ( GI (\fb, \ff) \cap GI (\gb, \gf) ) \\
  & = & (\lambda e_0.\, \bigjoin \{ d \mid \exists e.\, e \leq e_0
         \con d \leq \fb(e) \con e \leq \ff(d)
         \con d \leq \gb(e) \con e \leq \gf(d), \\
  & &   \lambda d_0.\,
         \bigjoin \{ e \mid \exists d.\, d \leq d_0
         \con d \leq \fb(e) \con e \leq \ff(d)
         \con d \leq \gb(e) \con e \leq \gf(d) \\
  & = & (\lambda e_0.\, \fst (\nu (d,e).\, (\fb(e) \meet \gb(e), e_0 \meet \ff(d) \meet \gf(d))), \\
  & &     \lambda d_0.\, \snd (\nu (d,e).\, (d_0 \meet \fb(e) \meet \gb(e), \ff(d) \meet \gf(d))))   
\end{eqnarray*}
or we can construct
\[
  (\fb, \ff) \meet (\gb, \gf)
    = (\fb \meet \gb, \ff \meet \gf)
\]  


We can prove $GI$ to be monoidal wrt. $\caplift$. $GI$ is lax monoidal
wrt. $\caplift$ just from being right adjoint to $HF$. To see that
$GI$ is oplax monoidal wrt. $\caplift$, we need to use that a relation
in the image of $GI$ iff it is join-closed and butterfly and also that
the intersection of join-closed closed butterfly relations is
a join-closed butterfly relation. Then the argument is like above for $G$.


We can prove $GI$ to be monoidal wrt.\ $\meet$ as follows.

We have $GI ((\fb,\ff) \meet g) \subseteq GI (f)$ and
$GI ((\fb,\ff) \meet (\gb,\gf)) \subseteq GI (\fb,\ff)$ from monotonicity of $GI$, therefore
$GI ((\fb,\ff) \meet (\gb,\gf)) \subseteq GI (\fb,\ff) \cap GI (\gb,\gf)$.

Now write $\kb$ for $\fb \meet \fb$ and $\kf$ for $\ff \meet \gf$. We have
\begin{eqnarray*}
  \fst ((\fb,\ff) \caplift (\gb,\gf)) & = & \lambda e_0.\, \fst (\nu (d,e).\, (\kb (e), e_0 \meet \kf (d))) \\
& = & \lambda e_0.\, \kb (\snd ( (\nu (d,e).\, (\kb(e), e_0 \meet \kf(d))) )) \\
& = & \lambda e_0.\, \kb (e_0 \meet \kf (\fst (\nu (d,e).\, (\kb (e), e_0 \meet \kf (d))))) \\
& \leq & \lambda e_0.\, \kb (e_0) \\
& = & \fb \meet \gb
\end{eqnarray*}
and similarly $\snd ((\fb,\ff) \caplift (\gb,\gf)) \leq \ff \meet \gf$, therefore
$(\fb,\ff) \caplift (\gb,\gf) \leq (\fb,\ff) \meet (\gb,\gf)$.

Applying $GI$, which is monotone, we get
$GI (\fb,\ff) \cap GI (g) \subseteq GI ((\fb,\ff) \caplift (\gb,\gf))
\subseteq GI ((\fb,\ff) \meet (\gb,\gf))$.

A more modular argument would prove that $I$ is monoidal wrt.\ $\meet$
and $\meet$.  Since we already know that $G$ is monoidal wrt.\
$\meet$, $GI$ is monoidal wrt.\ $\meet$.



% All posets we've considered have binary meets, so are lower
% semilattices. %(In fact they are complete lattices, but for now, this
% %is not important.)

% %Now assume $D$ and $E$ are \emph{frames}, i.e., have finite meets
% %distribute over arbitrary joins.

% We interpret nondeterministic choice between two programs as a binary
% meet.

% The poset maps we considered all preserve binary meets, so are lower
% semilattice maps. %are frame maps.

% \begin{eqnarray*}
%   F (R \cap R')
%   & = & \lambda p_0.\, \bigjoin \{ p \mid p \leq p_0 \con p \in R \cap R' \} \\
%   & = & \lambda p_0.\, \bigjoin (\{ p \mid p \leq p_0 \con p \in R\}
%         \cap \{ p \mid p \leq p_0 \con p \in R'\}) \\
%   & = & \lambda p_0.\, (\bigjoin \{ p \mid p \leq p_0 \con p \in R\})
%         \sqcap (\bigjoin \{ p \mid p \leq p_0 \con p \in R'\}) \\
%   & = & F(R) \sqcap F (R')
% \end{eqnarray*}        
  
% This uses $\bigjoin (S \cap S') = (\bigjoin S) \sqcap (\bigjoin S')$.
% $\leq$ is trivial from $T \subseteq U$ implying
% $\bigjoin T \leq \bigjoin U$.  For $\geq$, we need 
% $\bigjoin (S \cap S') \leq q$, i.e.,
% $\forall p.\ p \in S \con p \in S' \imp p \leq q$, to imply
% $(\bigjoin S) \sqcap (\bigjoin S') \leq q$ for any $q$.
% We have $(\bigjoin S) \sqcap (\bigjoin S') = \bigjoin \{ p \sqcap p' \mid
% p \in S \con p' \in S' \} \leq ??? \leq q$. ???

% I don't think this works even if we assume $D$, $E$ to be frames.
% So we only have $F (R \cap R') \leq F(R) \sqcap F(R')$???


\subsection*{Sequential composition}

\paragraph{Relations}

On $(\Pow(D \times D), \subseteq)$ we have a monoidal operation
$\bowtie$, more generally we can see it as an operation
$(\Pow(C \times D), \subseteq) \times (\Pow(D \times E), \subseteq)
\to (\Pow(C \times E), \subseteq)$:
\[
R \bowtie R' = \{ (c,e) \mid \exists d.\, (c,d) \in R \con (d, e) \in R' \}
\]  
This is our reference interpretation of sequential composition in our
reference model.

\bigskip

In the following, we will extensively use that
$\bigjoin R = (\bigjoin R|_0, \bigjoin R|_1)$.

\bigskip

The operator $\bowtie$ restricts to join-closed relations:
If $R$, $R'$ are join-closed, then $R \bowtie R'$ is also join-closed.

Proof: Given $S \subseteq R \bowtie R'$, we need to prove
$\bigjoin S \in R \bowtie R'$.

Define relations $S^*$ on $C \times D \times E$,
$S_0$ on $C \times D$ and $S_1$ on $D \times E$ by
\begin{eqnarray*}
  S^* & = & \{ (c, d, e) \mid (c, e) \in S \con (c, d) \in R \con (d, e) \in R' \} \\
  S_0 & = & \{ (c, d) \mid \exists e.\, (c, e) \in S \con (c, d) \in R \con (d, e) \in R' \} \\
  S_1 & = & \{ (d, e) \mid \exists c.\, (c, e) \in S \con (c, d) \in R \con (d, e) \in R' \}
\end{eqnarray*}  

We notice that $S_0 \subseteq R$, therefore $\bigjoin S_0 \in R$ by
join-closedness of $R$. Likewise $S_1 \subseteq R'$, therefore
$\bigjoin S_1 \in R'$ by join-closedness of $R'$.

This gives us
$(\prC (\bigjoin S^*), \prD (\bigjoin S^*)) = \prCD (\bigjoin S^*) =
\bigjoin S_0 \in R$ and
$(\prD (\bigjoin S^*), \prE (\bigjoin S^*)) = \prDE (\bigjoin S^*) =
\bigjoin S_1 \in R'$.
So $(\prC (\bigjoin S^*), \prE (\bigjoin S^*) \in R \bowtie R'$.

From $S \subseteq R$ we get that
$\exists d.\, (c, e) \in S \con (c, d) \in R \con (d, e) \in R'$ iff
$(c, e) \in S$. Therefore
$\bigjoin S = \prCE (\bigjoin S^*) = (\prC (\bigjoin S^*), \prE
(\bigjoin S^*)) \in R \bowtie R'$.


\paragraph{Functions on pairs}

Like for $\cap$ before, we define
\[
f \bowtielift g = F (G f \bowtie G g) 
\]  

Like before, from the adjunction $F \dashv G$, we get that
$G$ is lax monoidal wrt.\ $\bowtielift$.

Also like before, from a relation being in the image of $G$ iff it is
join-closed and from join-closedness being closed under $\bowtie$, we
also get that $G$ is oplax monoidal, so monoidal.

So $\bowtielift$ is a precise sound abstraction of $\bowtie$.

\bigskip

Explicitly, $\bowtielift$ is
\begin{eqnarray*}
f \bowtielift g
  & = & F (G (f) \bowtie G (g)) \\
  & = & \lambda p_0.\, \bigjoin \{ p \mid p \leq p_0 \con p \in G (f) \bowtie G (g) \} \\
  & = & \lambda (c_0,e_0).\,
        \bigjoin \{ (c,e) \mid (c,e) \leq (c_0,e_0)
        \con \exists d.\, (c,d) \leq f (c,d) \con (d,e) \leq g(d,e) \} \\
  & = & \lambda (c_0,e_0).\,
        \prCE (\bigjoin \{ (c,d,e) \mid (c,e) \leq (c_0,e_0)
        \con (c,d) \leq f (c,d) \con (d,e) \leq g(d,e) \}) \\
  & = & \lambda (c_0,e_0).\,
        \prCE (\nu (c,d,e).\, (c_0 \meet \prC(f(c,d)),
              \prD(f(c,d)) \meet \prD(g(d,e)), e_0 \meet \prE(g(d,e))))
\end{eqnarray*}  

\bigskip

Here is an alternative to $\bowtielift$. We define
\[
  f \boxtimes g = \lambda (c_0,e_0).
  \mathsf{let~} d_0 = \nu d.\, \prD(f(c_0,d)) \meet \prD(g(d,e_0))
  \mathsf{~in~} (\prC(f(c_0,d_0)), \prE(g(d_0,e_0)))         
\]  

\bigskip

We show that $\boxtimes$ is oplax monoidal, i.e.,
\[
G (f \boxtimes g) \subseteq G(f) \bowtie G(g)
\]

Proof: It is the case that, for all $c$, $e$, letting
$d_0 = \nu d.\, \prD(f(c_0,d)) \meet \prD(g(d,e_0))$, one has
\[
(c,e) \leq (\prC(f(c_0,d_0)), \prE(g(d_0,e_0))) 
\imp 
\exists d.\, (c,d) \leq f(c,d) \con (d,e) \leq g(d,e)
\]  
Indeed, take $d := d_0$ and from
$(c,e) \leq (\prC(f(c,d_0)), \prE(g(d_0,e)))$ and from $d_0$ being a
postfixpoint, i.e., $d_0 \leq \prD(f(c,d_0)) \meet \prD(g(d_0,e))$,
conclude $(c,d) \leq (\prC(f(c,d_0)),\prD(f(c,d_0))) = f(c,d_0)$ 
and $(d,e) \leq  (\prD(g(d_0,e)),\prE(g(d_0,e))) = g(d_0,e)$.

Consequently,
\begin{eqnarray*}
  G (f \boxtimes g)
  & = & \{ (c,e) \mid (c,e) \leq (f \boxtimes g)(c, e) \} \\
  & = & \{ (c,e) \mid (c,e) \leq \mathsf{let~} d_0 = \nu d.\, \prD(f(c_0,d)) \meet \prD(g(d,e_0))
  \mathsf{~in~} (\prC(f(c_0,d_0)), \prE(g(d_0,e_0))) \} \\
  & \subseteq & \{ (c,e) \mid
               \exists d.\, (c,d) \leq f(c,d) \con (d,e) \leq g(d,e) \} \\
  & = & G(f) \bowtie G(g)              
\end{eqnarray*}


\bigskip

We next show that $G$ is also lax monoidal wrt.\ $\boxtimes$, so
altogether monoidal. To do so, we first prove
$f \bowtielift g \leq f \boxtimes g$.

Proof: We need to show that for all $c_0,e_0$
\begin{eqnarray*}
\lefteqn{
        \mathsf{let~} (c^*,d^*,e^*) = \nu (c,d,e).\, (c_0 \meet \prC(f(c,d)),
  \prD(f(c,d)) \meet \prD(g(d,e)), e_0 \meet \prE(g(d,e)))) \mathsf{~in~} (c^*,e^*) } \\
& \leq &   
  \mathsf{let~} d_0 = \nu d.\, \prD(f(c_0,d)) \meet \prD(g(d,e_0))
  \mathsf{~in~} (\prC(f(c_0,d_0)), \prE(g(d_0,e_0)))  
\end{eqnarray*}

We first show that $d^* \leq d_0$ and then, using that, 
$c^* \leq \prC(f(c_0,d_0))$ and $e^* \leq \prC(f(c_0,d_0))$.

For $d^* \leq d_0$ we reason as follows:

For all $c$, $d$, $e$, we have
\[
  (c_0 \meet \prC(f(c,d)),
  \prD(f(c,d)) \meet \prD(g(d,e)),
  e_0 \meet \prE(g(d,e)))
  \leq
  (c_0, \prD(f(c,d)) \meet \prD(g(d,e)), e_0)
\]  
hence by monotonicity of $\nu$
\[
  \nu (c,d,e).\,  (c_0 \meet \prC(f(c,d)),
  \prD(f(c,d)) \meet \prD(g(d,e)),
  e_0 \meet \prE(g(d,e)))
  \leq
  \nu (c,d,e).\, (c_0, \prD(f(c,d)) \meet \prD(g(d,e)), e_0)
\]

Next we prove that
\[
  \nu (c,d,e).\, (c_0, \prD(f(c,d)) \meet \prD(g(d,e)), e_0) = (c_0, d_0, e_0)
\]
It is the case that $(c_0, d_0, e_0)$ is a postfixpoint of
$\lambda (c,d,e).\,  (c_0, \prD(f(c,d)) \meet \prD(g(d,e)), e_0)$: 
\[
(c_0, d_0, e_0) \leq (c_0, \prD(f(c_0,d_0)) \meet \prD(g(d_0,e_0)), e_0)
\]
since $d_0$ is by definition a postfixpoint of
$\lambda d.\, \prD(f(c_0,d)) \meet \prD(g(d,e_0))$.

But also, $(c_0, d_0, e_0)$ is greatest among these
postfixpoints. Indeed, if $(c',d',e')$ is some other postfixpoint,
i.e.,
\[
  (c',d',e') \leq (c_0, \prD(f(c',d')) \meet \prD(g(d',e')), e_0)
\]
then we do get $(c',d',e') \leq (c_0,d_0,e_0)$ since $c' \leq c_0$ and $e \leq e_0$ hold immediately from the assumption and we also have 
\[
  d' \leq \prD(f(c',d')) \meet \prD(g(d',e')) \leq \prD(f(c_0,d'))
  \meet \prD(g(d',e_0))
\]
so $d'$ is a postfixpoint of
$\lambda d.\, \prD(f(c_0,d)) \meet \prD(g(d,e_0))$, and so
$d' \leq d_0$ because $d_0$ is the greatest among them.

Putting everything together, we have
\begin{eqnarray*}
(c^*,d^*,e^*)
  & = &   \nu (c,d,e).\,  (c_0 \meet \prC(f(c,d)),
  \prD(f(c,d)) \meet \prD(g(d,e)),
  e_0 \meet \prE(g(d,e))) \\
  & \leq & 
  \nu (c,d,e).\, (c_0, \prD(f(c,d)) \meet \prD(g(d,e)), e_0) \\
  & = & (c_0,d_0,e_0)         
\end{eqnarray*}
so in particular $d^* \leq d_0$.

For $c^* \leq \prC(f(c_0,d_0))$ and $e^* \leq \prC(f(c_0,d_0))$, we
calculate:
\begin{eqnarray*}
c^* & = & c_0 \meet \prC(f(c^*,d^*)) \\
& \leq & \prC(f(c^*,d^*)) \\
& = & \prC(f(c_0 \meet f(c^*,d^*), d^*) \\
& \leq & \prC(f(c_0, d_0))  
\end{eqnarray*}  

\begin{eqnarray*}
e^* & = & e_0 \meet \prE(g(d^*,e^*)) \\
& \leq & \prE(g(d^*,e^*)) \\
& = & \prE(g(d^*, e_0 \meet g(d^*,e^*)) \\
& \leq & \prE(g(d_0, e_0))  
\end{eqnarray*}

This concludes the proof.

\bigskip

Using that $G$ is monoidal wrt.\ to $\bowtielift$ and monotone, we
conclude
\[
G f \bowtie G g = G (f \bowtielift g) \subseteq G (f \boxtimes g)
\]
i.e., $G$ is lax monoidal wrt.\ $\boxtimes$.


\paragraph{Backward and forward functions}



\paragraph{Backward functions only}





\subsection*{Iteration}

\end{document}